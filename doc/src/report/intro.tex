%!TEX root = report.tex
%

\chapter{Introduction}

Diderot is a parallel domain-specific language for programming image analysis and
visualization algorithms.
It supports a high-level programming model based on the mathematics of continuous
tensor fields.
These fields are reconstructed from discrete image-data sets (\eg{}, MRI data) using
separable convolution kernels.
We use \emph{tensors} to refer collectively to scalars,
vectors, and matrices, which encompasses the types of values produced
by the imaging modalities mentioned above, as well as values produced
by taking spatial derivatives of images.
Diderot permits programmers to express algorithms directly in terms of
tensors, tensor fields, and tensor field operations, using the same
mathematical notation that would be used in vector and tensor calculus
(such as $\nabla$ for the gradient).
Diderot is indended to be useful for prototyping image
analysis and visualization methods in contexts where a meaningful
evaluation of the methods requires its application to real image data,
but the real data volumes are of a size that requires efficient
parallel computation.
Diderot is also suited for educational contexts where the conceptual
transparency of the implementation is of primary importance.
In addition to providing a high-level mathematical programming model,
Diderot is also designed to be implemented on a range of parallel hardware,
including shared-memory multiprocessors and GPUs.

\section{Basic concepts}
The design of Diderot is based on three core concepts: image data, the continuous fields
that are reconstructed from them, and the strands that represent the computations over
the fields.
We give a high-level overview of these in this section.

\subsection{Images}

\subsection{Fields}
Computing with continuous tensor fields is one of the unique characteristics of Diderot.
Field values are constructed by convolving image data with
kernels (\CD{img${\color{\kwColor}\circledast}$bspln3}), but they can also be defined
by using higher-order operations, such as addition, subtraction, and scaling of fields.
Most importantly, Diderot supports differentiation of fields using the operators $\nabla$
(for scalar fields) and $\nabla\otimes$ (for higher-order tensor fields).
Two operations on fields are testing whether a point \CD{x} lies within the domain
of a field \CD{F} (\CD{inside(x, F)}) and \emph{probing} a field \CD{F} at a point \CD{x}
(\CD{F(x)}).
Probing and differentiating are fundamental to extracting geometric
information from fields.

\subsection{Strands}
The class of applications that Diderot targets are characterized as consisting of many
largely independent subcomputations.
For example, the rays in a volume renderer, the paths from fiber tractography, and
the particles in a particle system.
In Diderot, these mostly independent computations are modeled as \emph{strands}, which
execute in a \emph{bulk synchronous}
fashion~\cite{valiant:bridging-model-for-parallel,q-n-a-about-bsp}.


\section{Diderot by example}
% include VR-lite example here
\begin{figure}[p]
  \begin{quote}
    \lstset{language=Diderot}
    \input{vr-lite}
  \end{quote}%
  \caption{A simple volume renderer in Diderot}
  \label{fig:vr-lite}
\end{figure}%
