%!TEX root = report.tex
%

%BEGIN LATEX
\setlength{\textwidth}{6in}
\oddsidemargin 0.25in
\evensidemargin 0.25in
\addtolength{\textheight}{1.0in}
\addtolength{\topmargin}{-0.5in}
\parskip 5pt
%END LATEX

\usepackage{times}
%BEGIN LATEX
%-------------------------
% the following magic makes the tt font in math mode be the same as the
% normal tt font (i.e., Courier)
%
\SetMathAlphabet{\mathtt}{normal}{OT1}{pcr}{m}{n}
\SetMathAlphabet{\mathtt}{bold}{OT1}{pcr}{bx}{n}
%-------------------------
%END LATEX

%BEGIN LATEX
\usepackage{hevea}
\usepackage{grammar}
\renewcommand{\term}[1]{\textsc{#1}}
%END LATEX

\usepackage[newpage,toc]{manpage}
\usepackage{graphicx}
\usepackage{amssymb}
\usepackage{amsmath}
\usepackage{stmaryrd}

\usepackage{color}
\definecolor{Red}{rgb}{0.9,0.0,0.0}  % fixme
\definecolor{Green}{rgb}{0.0,0.4,0.0}
\definecolor{Blue}{rgb}{0.0,0.0,0.9}
\definecolor{DarkBlue}{rgb}{0.0,0.0,0.75}
\definecolor{Midnight}{rgb}{0.0,0.0,0.5}
\definecolor{Purple}{rgb}{0.5,0.0,0.4}
\definecolor{Black}{rgb}{0.0,0.0,0.0}
\definecolor{Yellow}{rgb}{1.0,1.0, 0.25}
\definecolor{Cyan}{rgb}{0.25,1.0, 1.0}

\newcommand{\cdColor}{Black}
\newcommand{\kwColor}{DarkBlue}
\newcommand{\comColor}{Red}

% code listings
%
\usepackage{listings}
\lstset{
  basicstyle=\ttfamily\scriptsize\color{\cdColor},
  keywordstyle=\color{\kwColor}\bfseries,
  commentstyle=\color{\comColor}\itshape}
\lstdefinelanguage{Diderot}{%
  otherkeywords={|,\#},
  morekeywords={%
    bool,%
    die,%
    else,%
    false,field,function,%
    identity,if,image,in,inf,initially,input,int,%
    kernel,%
    nan,new,%
    output,%
    real,%
    stabilize,strand,string,%
    tensor,true,%
    update,%
    vec2,vec3,vec4,%
    zeros},
  morekeywords=[2]{D},
  sensitive,%
  morecomment=[s]{/*}{*/},%
  morecomment=[l]//,
  morestring=[b]"}%

\lstset{
  language=Diderot
}

\newcommand{\CD}[1]{\textcolor{\cdColor}{\texttt{#1}}}
\newcommand{\KW}[1]{\textcolor{\kwColor}{\kw{#1}}}
\newcommand{\MKW}[1]{\textcolor{\kwColor}{\mkw{#1}}}

\newcommand{\appref}[1]{Appendix~\ref{#1}}
\newcommand{\chapref}[1]{Chapter~\ref{#1}}
\newcommand{\secref}[1]{Section~\ref{#1}}
\newcommand{\tblref}[1]{Table~\ref{#1}}
\newcommand{\figref}[1]{Figure~\ref{#1}}
\newcommand{\pref}[1]{{page~\pageref{#1}}}
\newcommand{\defref}[1]{Definition~\ref{#1}}
\newcommand{\lemmaref}[1]{Lemma~\ref{#1}}
\newcommand{\thmref}[1]{Theorem~\ref{#1}}

\newcommand{\eg}{{\em e.g.}}
\newcommand{\cf}{{\em cf.}}
\newcommand{\ie}{{\em i.e.}}
\newcommand{\etc}{{\em etc.\/}}
\newcommand{\naive}{na\"{\i}ve}
\newcommand{\ala}{{\em \`{a} la\/}}
\newcommand{\role}{r\^{o}le}

%
% font commands
\providecommand{\bftt}[1]{{\ttfamily\bfseries{}#1}}
\providecommand{\ittt}[1]{{\ttfamily\itshape{}#1}}
\providecommand{\kw}[1]{\bftt{\color{Purple}#1}}
\providecommand{\nt}[1]{{\rmfamily\itshape{#1}}}
\providecommand{\term}[1]{{\sffamily{#1}}}
\providecommand{\tyvar}[1]{#1}
\providecommand{\comment}[1]{#1}
\providecommand{\literal}[1]{#1}
%
% math-mode versions
\providecommand{\mkw}[1]{\ensuremath{\text{\kw{#1}}}}
\providecommand{\mnt}[1]{\ensuremath{\text{\nt{#1}}}}
\providecommand{\mterm}[1]{\ensuremath{\text{\term{#1}}}}

% braces
\newcommand{\LCB}{\sym{\char`\{}}
\newcommand{\RCB}{\sym{\char`\}}}

% special symbols
\newcommand{\DS}{\sym{\$}}
\newcommand{\PCT}{\sym{\%}}
\newcommand{\HASH}{\sym{\#}}
\newcommand{\BS}{\sym{\char`\\}}
\newcommand{\US}{\sym{\char`\_}}

% example code
%BEGIN LATEX
\newenvironment{EXAMPLE}{\begin{quote}\begin{lstlisting}}{\end{lstlisting}\end{quote}}
%END LATEX
%HEVEA \newenvironment{EXAMPLE}{\begin{alltt}}{\end{alltt}}
%HEVEA \usepackage{alltt}

%%%%%
% Some common math notation
%

% double brackets
\newcommand{\LDB}{\ensuremath{[\mskip -3mu [}}
\newcommand{\RDB}{\ensuremath{]\mskip -3mu ]}}

\newcommand{\dom}{\ensuremath{\mathrm{dom}}}
\newcommand{\rng}{\ensuremath{\mathrm{rng}}}

% sets
\newcommand{\SET}[1]{\ensuremath{\{#1\}}}
\newcommand{\Fin}{\textrm{Fin}}     % finite power set
\newcommand{\DISJOINT}[2]{\ensuremath{#1 \pitchfork #2}}
\newcommand{\finsubset}{\mathrel{\stackrel{\textrm{fin}}{\subset}}}

% finite maps
\newcommand{\finmap}{\mathrel{\stackrel{\textrm{fin}}{\rightarrow}}}
\newcommand{\MAP}[2]{\SET{#1 \mapsto #2}}
\newcommand{\EXTEND}[2]{\ensuremath{#1{\pm}#2}}
\newcommand{\EXTENDone}[3]{\EXTEND{#1}{\MAP{#2}{#3}}}
\newcommand{\SUBST}[3]{\ensuremath{#1[#2\mapsto{}#3]}}
\newcommand{\SUBSTTWO}[5]{\ensuremath{#1[#2\mapsto{}#3,#4\mapsto{}#5]}}

% typing judgments
%
\newcommand{\HasTy}[3]{#1 \vdash #2 : #3}
\newcommand{\UnopTy}[4]{\HasTy{#1}{#2}{#3 \rightarrow #4}}
\newcommand{\BinopTy}[5]{\HasTy{#1}{#2}{#3 \times #4 \rightarrow #5}}

% inference rules
\newcommand{\infer}[2]{\frac{\;{#2}\;}{\;{#1}\;}}
%
% labeled inference rule:
%   \INFER{name}{label}{conclusion}{assumption}
%
\newcommand{\INFER}[2]{%                                                      
  \begin{equation*}
    \infer{#1}{#2}                                                            
  \end{equation*}}                                                               

% natural numbers
%
\newcommand{\Nat}{\mathcal{N}}

% environments for type checking
%
\newcommand{\ENV}{\Gamma}

% Diderot types
%
\newcommand{\TYconst}{\iota}
\newcommand{\TYbool}{\mathbf{bool}}
\newcommand{\TYint}{\mathbf{int}}
\newcommand{\TYreal}{\mathbf{real}}
\newcommand{\TYrawten}[2]{\mathbf{rawten}\langle{}#1,#2\rangle{}}
\newcommand{\TYtensor}[1]{\mathbf{tensor}\langle{}#1\rangle{}}
\newcommand{\TYmatrix}[2]{\mathbf{matrix}\langle{}#1,#2\rangle{}}
\newcommand{\TYimage}[2]{\mathbf{image}_{#1}\langle{}#2\rangle{}}
\newcommand{\TYkern}[1]{\mathbf{kern}^{#1}}
\newcommand{\TYfield}[3]{\mathbf{field}^{#1}_{#2}\langle{}#3\rangle{}}
\newcommand{\TYvec}[1]{\mathbf{vec}_{#1}}

\newcommand{\Seq}[1]{\vec{#1}}

% notes
%BEGIN LATEX
\newcommand{\NOTE}[1]{%
  \par\leavevmode\noindent\textbf{[[ #1 ]]}\par\leavevmode\noindent}
%END LATEX
\newcommand{\CUT}[1]{}

%BEGIN LATEX
% timestamp
\newcount\timeHH
\newcount\timeMM
\timeHH=\time
\divide\timeHH by 60
\timeMM=\time
\count255=\timeHH
\multiply\count255 by -60 \advance\timeMM by \count255
\newcommand{\timestamp}{%
  \today{} ---
  \ifnum\timeHH<10 0\fi\number\timeHH\,:\,\ifnum\timeMM<10 0\fi\number\timeMM}
%END LATEX
%HAVEA \newcommand{\timestamp}{\today}

%
% A command to input code produced by extract-code.

\usepackage{ifthen}
\newcommand{\inputCode}[1]{%
  \ifthenelse{\boolean{hevea}}{\input{#1.hva}}{\input{#1.tex}}}
